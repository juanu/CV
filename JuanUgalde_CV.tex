% Juan A. Ugalde - Curriculum Vitae
%
% Copyright (C) 2004-2014 Jason R. Blevins
% Modified by Juan Ugalde, 2014.
% <jrblevin@sdf.org>
% http://jblevins.org/
%
% You may use use this document as a template to create your own CV
% and you may redistribute the source code freely.  No attribution is
% required in any resulting documents.  I do ask that you please leave
% this notice and the above URL in the source code if you choose to
% redistribute this file.

\documentclass[10pt,letterpaper]{article}

\usepackage{hyperref}
\usepackage{geometry}
\usepackage{etaremune}

% Fonts
%\usepackage[T1]{fontenc}
%\usepackage[urw-garamond]{mathdesign}

% Set your name here
\def\name{Juan A. Ugalde}

% The following metadata will show up in the PDF properties
\hypersetup{
  colorlinks = true,
  urlcolor = black,
  pdfauthor = {\name},
  pdfkeywords = {genomics, metagenomics, microbial ecology, comparative genomics},
  pdftitle = {\name: Curriculum Vitae},
  pdfsubject = {Curriculum Vitae},
  pdfpagemode = UseNone
}

\geometry{
  body={6.5in, 9.0in},
  left=1.0in,
  top=1.0in
}

% Customize page headers
\pagestyle{myheadings}
\markright{\name}
\thispagestyle{empty}

% Custom section fonts
\usepackage{sectsty}
\sectionfont{\rmfamily\mdseries\Large}
\subsectionfont{\rmfamily\mdseries\itshape\large}

% Other possible font commands include:
% \ttfamily for teletype,
% \sffamily for sans serif,
% \bfseries for bold,
% \scshape for small caps,
% \normalsize, \large, \Large, \LARGE sizes.

% Don't indent paragraphs.
\setlength\parindent{0em}

% Make lists without bullets and compact spacing
%\renewenvironment{itemize}{
%  \begin{list}{}{
%    \setlength{\leftmargin}{1.5em}
%    \setlength{\itemsep}{0.25em}
%    \setlength{\parskip}{0pt}
%    \setlength{\parsep}{0.25em}
%  }
%}{
%  \end{list}
%}

\begin{document}

% Place name at left
{\huge \name}

% Alternatively, print name centered and bold:
%\centerline{\huge \bf \name}

\bigskip

\begin{minipage}[t]{0.495\textwidth}
  Centro de Gen\'omica y Bioinform\'atica \\
  Facultad de Ciencias \\
  Universidad Mayor\\
  Campus Huechuraba, Camino a La Pir\'amide 5750 \\
  Huechuraba, Santiago\\
  Chile
\end{minipage}
\begin{minipage}[t]{0.495\textwidth}
  Phone: (+562) 23281472 \\
  Email: \href{mailto:juan@ecogenomica.cl}{juan@ecogenomica.cl} \\
  Homepage: \href{http://ugaldelab.github.io/}{http://ugaldelab.github.io/}
\end{minipage}

\section*{Education}

\begin{itemize}
  \item Ph.D. Marine Biology, University of California, San Diego, 2014.
    \begin{itemize}
    \item \emph{Dissertation:} ``Metagenomics approaches to unlock the microbial diversity of a hypersaline microbial ecosystem''.
    \end{itemize}

  \item Licenciatura in Molecular Biotechnology Engineering, Universidad de Chile. Santiago, Chile. 2007

\end{itemize}

\section*{Professional Experience}

\begin{itemize}

\item June 2014--present. Assistant Professor. Centro de G\'enomica y Bioinform\'atica, Facultad de Ciencias, Universidad Mayor. Santiago, Chile.

\item September 2008--May 2014. Graduate Student Researcher. Scripps Institution of Oceanography, University of California, San Diego. USA.

\item July 2004--August 2008. Project Engineer. Laboratory of Bioinformatics and Mathematics of the Genome. Facultad de Ciencias F\'isicas y Matem\'aticas, Universidad de Chile. Santiago, Chile.

\item January 2004--March 2004. Research assistant, group of Dr. Mikhail Matz. Whitney Laboratory, University of Florida. St. Augustine, FL. USA.

\item October 2003--December 2003. Project Engineer. Laboratory of Bioinformatics and Mathematics of the Genome. Facultad de Ciencias F\'isicas y Matem\'aticas, Universidad de Chile. Santiago, Chile.

\item January 2003--July 2003. Research assistant, group of Dr. Mikhail Matz. Whitney Laboratory, University of Florida. St. Augustine, FL. USA.

\item December 2001--October 2003. Research assistant. Laboratory of Bioinformatics and Gene Expression. INTA, Universidad De Chile. Santiago, Chile.


\end{itemize}


\section*{Fields of Research Interest}

Metagenomics, Microbial Genomics, Bioinformatics, Comparative Genomics, Environmental Microbiology.

\section*{Research}

\subsection*{Peer-Reviewed Publications}

\begin{etaremune}

\item Tang, X., Li, J., Mill\'an-Agui\~naga, N., Zhang, J.J., O'Neill, E., Ugalde, J.A., Jensen, P., Mantovani, S., Bradley, M. (2015). Identification of Thiotetronic Acid Antibiotic Bisynthetic Pathways by Target-directed Genome Mining. \textit{ACS Chemical Biology}. \textit{Accepted}.

\item Martin, L.J., Adams, R., Bateman, A., Bik, H.M., Haws, J., Hird, S.M., Hughes, D., Kembel, S.W., Kinney, K., Kolokotronis, S., Levy, G., McClain, C., Meadow, J.F., Medina, R.F., Mhuireach, G., Moreau, C.S., Munshi-South, J., Nichols, L., Palmer, C., Popova, L., Schal, C., Taeubel, M., Trautwein, M., Ugalde, J.A., Dunn, R.R. (2015). Evolution in the Indoor Biome. \textit{Trends in Ecology and Evolution}. 30(4):223-232.

\item Plominsky, A.M., Delherbe, N., Ugalde, J.A., Allen, E.E., Blanchet, M., Ikeda, P., Santiba\~nez, Hanselmann, K., Ulloa, O., De La Iglesia, R., von Dassow, P., Astorga, M., G\'alvez, M.J., Gonz\'alez., M.L., Henr\'iquez-Castillo, C., Vaulot, D., Lopes do Santos, A., van den Engh, G., Gimpel, C., Bertoglio, F., Delgado, Y., Docman, F., Elizondo-Patrone, C., Narv\'aez, S., Sorroche, F., Rojas-Herrera, M., Trefault, N. (2014). Metagenome Sequencing of the Microbial Community of a Solar Saltern Crystallizer Pond. \textit{Genome Announc.} 2(6): e01172-14. doi:10.1128/genomeA.01172-14.

\item Valenzuela, C., Ugalde, J.A., Mora, G.C., Alvarez, S., Contreras, I., Santiviago, C.A. (2014). Draft genome sequence of \textit{Salmonella enterica} serovar Typhi strain STH2370. \textit{Genome Announc.} 2(1): e00104-14. doi:10.1128/genomeA.00104-14.

\item Malfatti, F., Turk, V., Tinta, T., Mozetic, P., Manganelli, M., Samo, T., Ugalde J.A., Kovac, N., Stefanelli, M., Antonioli, M., Fonda-Umani, S., Del Negro, P., Cataletto, B., Hozic Zimmermann, A., Ivo\v{s}evi\'{c} Denardi, N., Žuti\'{c}, V., Svetli\v{c}i\'{c}, V., Mi\v{s}ic Radi\'{c},T., Radi\'{c}, T., Fuchs D., Azam, F. (2014). Microbial mechanisms coupling carbon and phosphorous cycles in phosphorous-limited northern Adriatic Sea, and response to phosphorous enrichment. \textit{Science of the Total Environment}. 470:1173-1183.

\item Ugalde, J.A., Narasingarao, P., Kuo, S., Podell, S., Allen, E.A. (2013). Draft Genome Sequence of “\textit{Candidatus} Halobonum tyrrellensis” Strain G22, Isolated from the Hypersaline Waters of Lake Tyrrell, Australia. \textit{Genome Announc.} 1(6):e01001-13. doi:10.1128/genomeA.01001-13.

\item Podell, S., Emerson, J., Jones, C.M., Ugalde, J.A., Welch, S., Heidelberg, K.B., Banfield, J.F., Allen, E.E. (2013). Seasonal Fluctuations in Ionic Concentrations Drive Microbial Succession in a Hypersaline Lake Community. \textit{ISME Journal.} Advance online publication, December 12, 2013; doi:10.1038/ismej.2013.221.

\item Kharbush, J., Ugalde J.A., Hogle, S., Allen, E.E., Aluwihare, L. (2013). Diversity of composite bacterial hopanoids and their microbial producers across oxygen gradients in the water column of the California Current. \textit{Applied and Environmental Microbiology}. Published ahead of print, doi:10.1128/AEM.02367- 13.

\item Ugalde, J.A., Gallardo, M.J., Belmar, C., Muñoz, P., Ruiz-Tagle, N., Ferrada- Fuentes, S., Espinoza, C., Allen, E.E., Gallardo, V.A. (2013). Microbial life in a Fjord: Metagenomic analysis of a microbial mat in Chilean Patagonia. \textit{PLoS ONE} 8(8): e71952

\item Podell, S., Ugalde, J.A., Narasingarao, P., Banfield, J.F., Heidelberg, K.B., Allen, E.E. (2013). Assembly-Driven Community Genomics of a Hypersaline Microbial Ecosystem. \textit{PLoS ONE} 8(4): e61692.

\item Ugalde, J.A., Podell, S., Narasingarao P., Allen, E.E. (2011). Xenorhodopsins, an enigmatic new class of microbial rhodopsins horizontally transferred between archaea and bacteria. \textit{Biology Direct} 6:52.

\item Narasingarao, P., Podell, S., Ugalde, J.A., Brochier-Armanet, C., Emerson, J.B., Brocks, J.J., Heidelberg, K.B., Banfield, J.F., Allen, E.E. (2011). De novo metagenomic assembly reveals abundant novel major lineage of Archaea in hypersaline microbial communities. \textit{ISME Journal.} Advance online publication, June 30, 2011; doi:10.1038/ismej.2011.78.

\item *Levican, G., *Ugalde, J.A., Ehrenfeld N., Maass, A., Parada, P. (2008). Comparative genomic analysis of carbon and nitrogen assimilation mechanisms in three indigenous bioleaching bacteria: predictions and validations. \textit{BMC Genomics} 9:581.

\item Chang, B.S., Matz, M.V., Ugalde, J.A. (2005). Applications of ancestral protein reconstruction in understanding protein function: GFP-like proteins. \textit{Methods in Enzynmology} 395: 652-670.

\item Matz, M.V., Labas, Y.A. and Ugalde, J. (2005). Evolution of function and color in GFP- like proteins. In: Green Fluorescent Protein: Properties, Applications and Protocols, 2nd Ed. (M. Chalfie and S.R. Kain eds.) Wiley press.

\item Ugalde, J.A., Chang, B.S., Matz, M.V. (2004). Evolution of coral colors recreated. \textit{Science} 305:1433.

\item Shagin, D.A., Barsova, E.V., Yanushevich, Y.G., Fradkov, A.F., Lukyanov, K.A., Labas, Y.A., Semenova, T.N., Ugalde, J.A., Meyer, A., Nunes, J.M., Widder, E.A., Lukyanov, S.A., Matz, M.V. (2004). GFP-like proteins as ubiquitous Metazoan superfamily: evolution of functional features and structural complexity. \textit{Molecular Biology and Evolution} 21:841-850.

\end{etaremune}

%\subsection*{Papers Under Review or Revision}

%\begin{itemize}

%\end{itemize}
\section*{Grants}

\begin{itemize}
\item Fondecyt Iniciaci\'on 11140666. \textit{Development of reproducible bioinformatics workflows for comparative microbial genomics.} 2014-2017. (Principal Investigator). 

\item Amazon Web Services (AWS) Research Grant. US\$15,000 for computer analysis using the AWS infraestructure. 2013-2015.

\item Instituto Ant\'artico Chileno (INACH). Project RG\_31\_15. \textit{Diversity and inter-annual variability of eukaryote microbial communities in Antarctic coastal waters.} 2016-2018. (Principal Investigator).


\end{itemize}


%\subsection*{Work in Progress}
%
%\begin{itemize}

%\end{itemize}

% \subsection*{Scientific Software}
%
% \begin{itemize}

% \end{itemize}


\section*{Fellowships, \& Awards}

\begin{itemize}


\item 2011. Zobell Fellowship Award. US\$2000
\item April 2011-June 2011. Latin American Studies Fellowship, Scripps Institution of Oceanography, UCSD.
\item March 2011. National Institute of Diabetes and Digestive and Kidney
Diseases (NIDDK)/National Institute on Aging (NIA) scholarship to attend the Keystone Symposia on Microbial Communities as Drivers of Ecosystem Complexity, March 25-30, 2011. Beaver Run Resort, Breckenridge, Colorado, USA.

\item September 2008-August 2012. Fulbright-Conicyt fellowship for graduate studies.
\item August 2004. Best oral presentation at the XXI Annual congress of the National Biochemistry Student Association, in Valdivia, Chile.
\item January-April 2003. Latin American Exchange Program. Funding for a research experience as an undergraduate student at the Whitney Laboratory of the University of Florida. The Whitney Laboratory and the Grass Foundation provided the funding.

\end{itemize}

\section*{Invited Talks}

\begin{itemize}

\item Coloquios de Microbiolog\'ia, August 25, 2014. Concepción, Chile.

\item I Simposio en Bioinform\'atica Integrativa, 16 Julio, 2014. Santiago, Chile.

\item Illumina Users Meeting. May 29, 2014. San Diego, California. 

\item XXXV Annual Meeting of the Chilean Microbiology Society. November, 26-30, 2013. Marbella, Chile.

\end{itemize}

 
\section*{Oral presentations}

\begin{itemize}
\item Ugalde, J.A., Podell, S., Banfield, J., Allen, E.E. High-resolution community genomics of a hypersaline microbial ecosystem. ISCB-Latin America, March 17- 21, 2012. Santiago, Chile.
\item Ugalde, J.A., Chang, B.W., Matz, M.V Reconstruction of ancestral proteins: Application in coral fluorescent proteins. XLVIII Anual Meeting of the Biological Society of Chile. October 13-16, 2005. Pucon, Chile.
\item Ugalde, J.A., Chang, B.W., Matz, M.V. Evolutionary reconstruction of coral colors. XXI Annual Congress of the National Biochemistry Student Association. August 25-28, 2004. Valdivia, Chile.
\item Ugalde, J.A., Kelmanson, I., Bulina, M., Bielawski, J.P., Matz, M.V. Natural Selection in fluorescent proteins from corals and corallimorphs. XI Annual Conference of the Society for Molecular Biology and Evolution. Junio 25-29, 2003. Newport Beach, CA. USA.
\item Ugalde, J.A., Mendez, M., Gonzalez, M. Global analysis of gene expresión in Saccharomyces cerevisiae: search for patterns of spatial and temporal correlations. II Workshop in Bioinformatics. October 16-17, 2003. Santiago, Chile.

\end{itemize}

%
%\section*{Professional Activities}
%
%\begin{itemize}
%\item Reviewer for
%  \textit{Econometrica}, % 2011 (1)
%  \textit{Economic Inquiry}, % 2008 (1)
%  \textit{Journal of Business and Economic Statistics}, % 2009 (1), 2013 (2)
%  \textit{Journal of Computational Finance}, % 2010 (1)
%  \textit{Journal of Econometrics}, % 2009 (1), 2011 (2), 2012 (1), 2013 (1)
%  \textit{Review of Economics and Statistics}, % 2013 (1)
%  \textit{Scandinavian Journal of Statistics}, % 2009 (1)
%  \textit{Pakistan Journal of Statistics}, % 2014 (1)
%  \textit{Quantitative Economics}, % 2012 (1)
%  Cambridge University Press, % 2011 (1)
%  National Science Foundation, % 2012 (1)
%  Polish National Science Centre. % 2014 (1)
%\item Departmental service:
%  Graduate Studies Committee (2012-13),
%  Faculty Recruiting Committee (2010-11, 2011-12),
%  Graduate Program Review Committee (2013-14),
%  G.S. Maddala Prize Committee (2011, 2012, 2013).
%\item Member, Econometric Society, 2006--Present.
%\item Member, American Economic Association, 2010--Present.
%\item Member, American Statistical Association, 2011--Present.
%\end{itemize}

\section*{Teaching}

\begin{itemize}

\item Co-instructor, Environmental Microbiology (Graduate level). Facultad de Ciencias, Universidad Mayor. 2014.

\item Teaching Assistant, Introduction to Computing at SIO (SIO232). Fall 2013.

\item Teaching Assistant, Environmental Biology (ESYS101). Fall 2012.

\item Co-instructor, Environmental Microbiology. July 2010. The class was taught as part of the Academic Connections program, at UC San Diego.

\item Teaching Assistant. Course \textit{Ecology and Diversity of Marine Microorganisms} (ECODIMM-VI). January 2010. Summer course for graduate and post-graduate students. Universidad de Concepcion, Dichato, Chile.

\end{itemize}

%\section*{Advising}

%\begin{itemize}
%\item Ph.D. Students, Primary Advisor (year, initial placement):
%  \begin{itemize}
%  \end{itemize}
%\item Ph.D. Students, Secondary Advisor (year, initial placement):
%  \begin{itemize}
%  \end{itemize}
%\item Undergraduate Honors Students:
%  \begin{itemize}
%  \end{itemize}
%\end{itemize}

% \section*{Conference Participation}

% \begin{itemize}

% \end{itemize}

%\section*{Miscellaneous}
%
%\begin{itemize}

%\end{itemize}

% Footer
\medskip
\begin{center}
  \begin{small}
    Last updated: \today
  \end{small}
\end{center}
%
\end{document}
